\section*{\hfil \ \ \ Abstract \hfil}
\addcontentsline{toc}{section}{Abstract}
One of the traditional ways to construct a web servers is to use
a pool of workers and a handler function that computes responses
for all types of requests.
This project introduces an alternative architecture in which 
servers are constructed as a network of communicating components.
This architecture is based on the idea that different requests 
can be processed using different single purpose components.
The main advantages of this approach are the clarity of the design 
and reusability of the components.

The project introduces a toolkit,
called \texttt{mpserver}, that implements this architecture as a 
package in the Go programming language. Implementation of basic and 
more advanced elements of the toolkit is explained in detail. 
One problem of servers implemented using this approach is that 
their throughput can be very limited.
To address this problem the toolkit implements 
various forms of load management that allow higher throughput. 
The functionality of the toolkit is showed by means of several
examples.