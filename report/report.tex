\documentclass[12pt,a4paper]{article}
\usepackage[utf8]{inputenc}
\usepackage{listings}
\usepackage{listings-golang} % import this package after listings
\usepackage{color}

\lstset{ % add your own preferences
    frame=single,
    basicstyle=\footnotesize,
    keywordstyle=\color{blue},
    numbers=none,
    showstringspaces=false, 
    stringstyle=\color{red},
    tabsize=4,
    language=Golang
}

\begin{document}
\section{Introduction}
The goal of this project is to construct a well-structured
library of communicating components, in the Go language, to support
the programming of a family of webservers that could range in scale from a full-scale commercial server to an embedded server used to provide a browser-based GUI to a specific application.

\subsection{Motivation}
The vast majority of web server frameworks have a single function that
computes a response to the client requests and a pool of workers that can
be used to execute this function. When a request comes in a worker is 
allocated from the pool and it executes the function.
\\
\\
This project examines an alternative approach, where the server is a network 
of communicating components. Each component is highly specialized and communicates
with other components using message passing.

The components are plugged together into a pipeline using channels.
The incoming request are fed into the input end of the pipeline and 
after passing through the whole network the response is written to the
client on the other end of the network. That is we can view the server
as a pipeline that transfers requests into responses.

\subsection{Introduction to Message passing Concurrency}

\subsection{Introduction to go}

\newpage
\section{Implementation}

mpserver is a go library that implements a framework for building 
concurrent web-servers based on message passing.

The web-server consists of a multiple Components which are plugged 
together using channels to form a pipeline that transforms requests 
to responses.

\subsection{Values} 
Channels between Components transfer values of the following type:

\begin{lstlisting}
type Value struct {
    Request *http.Request
    Writer http.ResponseWriter
    Result Any
    Done chan<- bool
    ResponseCode int
}
\end{lstlisting}
Where:
\begin{itemize}
  \item Request is the http request made by the client.
  \item Writer is an object that writes the response back to the 	client.
  \item Result is the result computed so far (initially nil).
  \item Done is a signaling channel. Signal should be send on this channel,
		after the response is written at the end of the pipeline.
  \item ResponseCode is the response code that should be returned to the client.
\end{itemize}

\subsection{Components}
Component is a generic part of the pipeline with an input and output channel.
The type Component is defined as follows:

\begin{lstlisting}
type Component func (in <-chan Value, out chan<- Value)
\end{lstlisting}

Every Component should satisfy the following:
\begin{itemize}
  \item When run Component should read values from the input channel
				and output processed values on the output channel while the input 
				channel is open.
	\item Component terminates normally only if its input channel is closed.
	\item Before terminating, every Component closes its output channel.
	\item Component can only close its output channel and it can only do so
				after its input channel have been closed.
	\item For every value that a Component reads from the input channel, 
				it should write a value to the output channel. So, that every request 
				gets a response.
  
\end{itemize}

\subsubsection{Making Components}
The package contains a helper function that constructs the components,
so that they follow the specification above. It's signature is below:
\begin{lstlisting}
func MakeComponent(f ComponetFunc) Component
\end{lstlisting}

The Component Function is a helper type with the following definition:
\begin{lstlisting}
type ComponetFunc func (val Value) Value
\end{lstlisting}
It is a function that takes a value and returns another value.

The Component constructed by MakeComponent function reads value from the
input channel, runs provided function f on this value and then outputs the
result of f.

Therefore anyone using the package just needs to define the function that
transforms the input value to the output value and then the MakeComponent
function will construct the actual component for them.

\subsubsection{Constant Component}
Constant Component is a simple component generator that takes a value c
of any type and returns a component that writes c to the Result field of all inputted values and then outputs them. It's signature is below.
\begin{lstlisting}
func ConstantComponent(c Any) Component
\end{lstlisting}

\subsection{Linking components}
The library provides a component generator Link Components that takes 
any number of components and returns a component that behaves as their 
linear combination. That is as a pipeline constructed from these components 
in the order in which they are provided. Its signature is below:
\begin{lstlisting}
func LinkComponents(components ...Component) Component
\end{lstlisting}
The subcomponents are linked and run when the returned component is run. 
The last of the provided components is run in the goroutine where
the combined component was run.

\subsection{Writers}
Writer is the end of the pipeline which writes results to the client.
The type Writer is defined as follows:

\begin{lstlisting}
type Writer func (in <-chan Value, errChan chan<- Value)
\end{lstlisting}

It is a function with input channel and channel for reporting errors.
\begin{itemize}
	\item When a Writer writes the result to the client, it should send a signal
				on done channel, so that the connection to client can be closed.
	\item The writer terminates when its input channel is closed. It shouldn't
			  close the error channel, as other processes might be using it.
\end{itemize}

\subsubsection{Error Writer}
Error Writer is a function with the following signature:

\begin{lstlisting}
func ErrorWriter(in <-chan Value)
\end{lstlisting}
It has only one input channel. The writer terminates when the channel
is closed. The channel should be closed only after all Writers that 
are using it terminated.

\subsection{Conditions and Splitter}
\subsubsection{Conditions}
Condition type is defined as follows:

\begin{lstlisting}
type Condition func (val Value) bool
\end{lstlisting}
It is a function that takes a value and returns a boolean.

\subsubsection{Splitter}
Conditions are used in Splitter, which is a component with the 
following signature:

\begin{lstlisting}
func Splitter(in <-chan Value, defOut chan<- Value, 
			  outs []chan<- Value, conds []Condition)
\end{lstlisting}
It takes an input channel, default output channel, array of output channels and array of conditions.
The number of output channels and the number of conditions should be the same.
For every input value the Splitter evaluates the conditions from first to last.
When a condition returns true the value is written to a corresponding output channel 
and the processing of the current value terminates. If all conditions return false
then the value is written to the default output channel.

\subsection{States and Session Management Component}
\subsubsection{States}
State is an interface defined as follows:

\begin{lstlisting}
type State interface {
    Next(val Value) (State, error)
    Terminal() bool
    Result() Any
}
\end{lstlisting}
\begin{itemize}
	\item The Next function returns the next state when provided with a value.
	\item The Terminal function indicates whether the current state is a terminal state.
	\item The Result function returns the result that possibly after further 
				processing should be returned to the client.
\end{itemize}

\subsubsection{Session management component}
Session management component is a following component generator.
\begin{lstlisting}
func SessionManagementComponent(
		initial State, seshExp time.Duration) Component
\end{lstlisting}
It takes an initial state for each session and a session expiration time and 
returns a Component that behaves as a session manager. 
\\
\\
The component behaves as follows:
\\
The component stores a mapping from session ids to current state of the session.

Every new request gets assigned a unique session id and a mapping from the generated 
id to the next state, which is generated from the initial state using the provided value,
is stored. The result of the call to the `Result` function on the next state with the provided 
value is then sent down the pipeline.

When a request with set `Session-Id` header comes in, the component gets the current 
state for the session and gets the next state for it based on the provided value.
The mapping is then updated with the generated state and the result from the current state
is passed down the pipeline.

When a session reaches a Terminal state, the session id is removed from the map
and the final result is outputted.

The states in the mapping are timestamped and are removed from the map after
the session expiration time, if they haven't finished before that.

The abstraction using states allows the users of the package to implement the
logic for the session, independently of the session management part.

\subsection{Caching}
Cache Component is a Component generator with the following signature:
\begin{lstlisting}
func CacheComponent(worker Component, 
					expiration time.Duration) Component
\end{lstlisting}
The returned component behaves as follows:
\begin{itemize}
	\item When a value is inputted that the component hasn't seen before, then
				the value is passed to the provided worker and the result is stored in
				a map.
	\item When the component gets a previously seen value, it returns the value
			  stored in the map, if it hasn't expired yet. If it expired, then the component
				treats it as a new value.
	\item Expired values are regularly deleted from the map.
\end{itemize}

For the generated component to function properly, the worker must output
a value for every value that is sent to it.

\subsection{Load Balancing}
Load balancing component a Component generator with the following signature:
\begin{lstlisting}
func LoadBalancingComponent(addTimeout, 
		removeTimeout time.Duration, worker Component) Component
\end{lstlisting}
The returned component has an array of workers which can be easily shut down.
The initial number of workers is 1. 
\\
\\
For every inputted value, the component behaves as follows:
\begin{itemize}
	\item It tries to send the value to the workers, if this is not successful, 
				before the addTimeout, then the component creates a new worker and then
				tries to send the value to the workers again. This will be successful as
				there is at least one worker that is not busy.
	\item The workers send their results further down the pipeline.
	\item If there are no incoming values for removeTimout time, then the component
				shuts down one of the workers, if there is more than one worker.
	\item When the input channel of the component is closed, then it shuts down all
				the workers, then it closes its output channel and terminates.

\end{itemize}

\subsection{Error handling}
\subsubsection{Error Passer}
Error Passer is a component generator with the following signature:

\begin{lstlisting}
func ErrorPasser(worker Component) Component
\end{lstlisting}
The generated component behaves as follows:
\begin{itemize}
	\item For every inputted value it checks whether the result is an error
				and if that is the cases it just outputs the value.
	\item If the result of the inputted value is not an error, then the component
				passes the value to the worker, which after processing it passes 
				it further down the pipeline.
\end{itemize}

\subsubsection{Panic handling component}
Panic handling component is a component generator with the following signature:

\begin{lstlisting}
func PannicHandlingComponent(worker Component) Component
\end{lstlisting}
It takes a component that can cause panic and returns a component the behaves as follows:
\begin{itemize}
	\item It passes every inputted value to the worker and gets the result from it and then passes
				the result further down the pipeline.
	\item In case the worker crashes, that is causes panic, the component writes error as a result 
				for the value that caused the crash and the restarts the worker.
	\item The component can itself cause a panic if the worker closes its input channel, or if 
				it closes its output channel before its input channel is closed. That is if the worker
				violates some of the basic requirements for Components.
\end{itemize}

\subsection{Network Components}
\subsubsection{Request Copier}
Request Copier is a component generator with the following signature:
\begin{lstlisting}
func RequestCopier(scheme, host string) Component
\end{lstlisting}
It takes string parameters scheme (e.g. http) and host (e.g. www.google.com) 
and returns a component that for each inputted value copies the Request made 
by the client to the Result field and updates it with the given scheme and 
path and then outputs this value.

\subsubsection{Network Component}
Network Component is a component generator with the following signature:
\begin{lstlisting}
func NetworkComponent(client *http.Client) Component
\end{lstlisting}
It takes a single parameter, which is an http client.
The generated component for each inputted value makes an http request,
taken from the Result field of the value, using the provided http client.
If there is no request in the Result field or the request fails, then
the component returns an appropriate error. 

The result of the request is fetched in a new goroutine, so slow requests
won't block the pipeline.

\subsubsection{Response Processor}
The response type of the mpserver package is defined as a following struct:
\begin{lstlisting}
type Response struct {
		Header http.Header
		Body []byte
}
\end{lstlisting}
It holds the header and body of a http response. The difference between
this definition and the definition in the default http package is that
the body of my response is a slice of bytes. That is the body have already been
read and the connection to server has been closed.
\\
\\
The Response Processor is a component that for each inputted value
looks if provided Result is the default http Response, and if it is
it then reads it and transfers it to mpsever response type, which it
then writes to the Result field and outputs the value. If no response
is provided, then the component outputs an appropriate error.

To use this component after the Network component, the Response Processor
should be wrapped in an error passing component, so that when the 
Network component fails, then the response processor won't have to do
any work.

\subsubsection{Proxy Component}
Proxy component is a combination of Request Copier, Network component 
and Response processor in a linear pipeline. It acts as a proxy server
in the following sense. It forwards the provided request to the specified
host and then reads and outputs the response. 

All components are wrapped
in an Error passer to avoid doing unnecessary work and to make the component
more robust. Its implementation is below:
\begin{lstlisting}
func ProxyComponent(scheme, host string, 
					client *http.Client) Component {
	return LinkComponents(
		ErrorPasser(RequestCopier(scheme, host)),
		ErrorPasser(NetworkComponent(client)),
		ErrorPasser(ResponseProcessor))
}
\end{lstlisting}

\newpage
\section{Examples}
\subsection{Hello world! server}
Below is an example of a simple server that responds with a
simple string "Hello world!". It's main parts are ConstantComponent,
which writes the string "Hello world!" to the Result field of each request and 
StringWriter, that writes a string response back to the client. 

\begin{lstlisting}
package main

import(
    "log"
    "net/http"
    "mpserver"
)

func main() {
    in := make(mpserver.ValueChan)
    out := make(mpserver.ValueChan)
    errChan := make(mpserver.ValueChan)

    go mpserver.ConstantComponent("Hello world!")(in, out)
    go mpserver.StringWriter(out, errChan)
    go mpserver.ErrorWriter(errChan)

    mux := http.NewServeMux()
    mpserver.Listen(mux, "/hello", in)
    log.Println("Listening on port 3000...")
    http.ListenAndServe(":3000", mux)
}
\end{lstlisting}

\section{Experiments}
\section{Conclusion}


\end{document}




