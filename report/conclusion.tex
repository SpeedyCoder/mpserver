\section{Conclusion}
\label{sec:conclusion}
\subsection{Summary}
I successfully developed a compositional toolkit for construction of 
web-servers in Go. I first introduced the idea behind the toolkit that 
makes the construction of web-servers very intuitive and straight forward. 

Then I presented my implementation. I started by introducing the basic types
and then went on to describe much more complicated components and component generators
that make it easy to implement complex behaviors. I demonstrated the usage of the 
toolkit on a handful of simple examples and performed a case study of building
a more complex server using the toolkit. This showed that the toolkit is
functional and simple to use. 

% TODO: talk about testing

\subsection{Critical analysis}
As noted above the toolkit is easy to use due the fact that response 
generation diagrams translate directly to program code. The components
in the toolkit are very general and highly configurable. The toolkit
also gives programmers control over how many resources are used for 
different tasks. 

However, the performance of servers implemented in the toolkit can be 
very poor if they are not configured properly. This is because when using a simple
pipeline without any load balancing all requests must go through all parts
of the server one by one, limiting the throughput. The provided
tools for load balancing mitigate this problem.

As one would expect the performance of the web-server
is worse than just using the standard library. This is due to overheads
for the usage of channels, creating the \texttt{Job} objects
and switching between different goroutines.

On the other hand the toolkit provides a lot of functionality over the
standard library and makes implementing servers more straight forward. 
Hence, developing servers using the toolkit should be
significantly easier and therefore quicker. So, if a small performance
penalty is acceptable, then my toolkit would be the preferred alternative.

\subsection{Future Work}
To make the toolkit production-ready I would perform more testing of all
components, implement more components and improve my documentation
of the toolkit. 

One might want to implement a version of the \texttt{Storage}
that uses an SQL database or a component that accesses a database and 
returns the requested data. As noted before, other strategies 
for load balancing can be implemented. 
They can possibly be abstracted into their own type, so that the
load manager can be parameterized based on our needs.

The library is very general and extensible. Hence, if we wanted to support 
other protocols such as webSocket we can easily implement a different 
type of \texttt{job} and associated components and writers in similar fashion.

