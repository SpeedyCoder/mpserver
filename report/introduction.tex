\section{Introduction}
In this first chapter I introduce the motivation behind this project and 
give an overview of my design. I also review related work done on similar
problems and give a brief summary of this report. 

\subsection{Motivation}
The vast majority of web server frameworks such as Apache
use a single handler function that
computes responses to all types of client requests. When a request comes in, a 
worker is allocated from a pool of workers and it then executes the handler function
and returns the result to the client.

This project examines an alternative approach to building web servers. 
I propose to construct web servers as networks of highly specialized 
communicating components. The communication should be done using
message passing. This builds on the premise that we can deal
with different types of requests using different single purpose components.

The main advantages of the proposed architecture are that it is easy to understand
and configure a server. A conceptual design of response generation from a request translates 
directly into a server architecture. That is a data flow diagram 
representing a transformation
of a request into a response can be directly translated into program code.
Due to the compositional nature of this design its also very easy to reuse code
and to use wrappers to introduce new behaviors such as caching of responses.
Note that wrappers in this project wrap around the input and output channels
of a component rather than the component itself.

\subsection{Proposed Architecture}
The overall server architecture can be explained as follows.
To implement complex behaviors the components, each representing 
a single action, are plugged together into a pipeline using channels. 
The incoming
request together with the result computed so far and other 
information needed for computation and writing the result back to the client
are passed through the channels.

The incoming requests are fed into the input end of the network of pipelines and 
after passing through the network the response is written to the
client on the other end. Hence, we can view this server
as a network of pipelines that transform requests into responses.

To implement this architecture I have constructed a package in the go programming 
language called \texttt{mpserver}, that allows the construction of such web servers.
It uses the built in \texttt{http} package to handle the details of
the HTTP \cite{http} protocol. The toolkit itself concentrates on the 
structure of the web-server. Figures \ref{fig:exampleServer1} and 
\ref{fig:exampleServer2} below show designs of two servers and their parts
using the toolkit. Their behaviors are explained in later chapters.

\begin{figure}[h]
\centering
\begin{tikzpicture}[node distance = 2cm, auto]
    % Nodes
    \node [listener] (listener) {Listener};
    \node [component, right of=listener, node distance=3cm] (getter) {File Getter};
    \node [component, right of=getter, node distance=3cm] (splitter) {Router};
    \node [writer, right of=splitter, node distance=3cm] (fwriter) {File Writer};
    \node [writer, above of=fwriter] (gwriter) {Gzip Writer};
    % Edges
    \path [line] (listener) -- (getter);
    \path [line] (getter) -- (splitter);
    \path [line] (splitter) -- (fwriter);
    \path [line] (splitter) |- node[condition, cross]{} (gwriter);
\end{tikzpicture}
\caption[scale=1.0]{File server with compression.}
\label{fig:exampleServer1}
\end{figure}

\begin{figure}[h]
\centering
\begin{tikzpicture}[node distance = 1.5cm, auto,
    workerWidth/.style={text width=5em, text centered},
    workerHeight/.style={minimum height=4em},
    UWidth/.style={minimum width=0.6cm},
    UHeight/.style={minimum height=0.6cm}
    ]
    % Nodes
    \node [component] (worker1) {Worker};
    \node [workerWidth, UHeight, below of=worker1] (passer1) {};
    \node [workerWidth, UHeight, above of=worker1] (passer2) {};
    \node [component, above of=passer2] (worker2) {Worker};
    \node [component, below of=passer1] (worker3) {Worker};
    \node [workerWidth, UHeight, below of=worker3] (passer3) {};

    % Left
    \node [UWidth, workerHeight, left of=worker1, node distance=2cm] (left1) {};
    \node [UWidth, workerHeight, left of=worker2, node distance=2cm] (left2) {};
    \node [UWidth, workerHeight, left of=worker3, node distance=2cm] (left3) {};

    \node [UWidth, workerHeight, left of=left1, node distance=1.5cm] (lleft1) {};
    \node [UWidth, workerHeight, left of=left2, node distance=1.5cm] (lleft2) {};
    \node [UWidth, workerHeight, left of=left3, node distance=1.5cm] (lleft3) {};

    \draw[fill=gray!12] (lleft2.north west) -- (lleft2.west |- passer3.south) --
          (lleft2.east |- passer3.south) -- (lleft2.north east) --
          cycle;

    % Right
    \node [UWidth, workerHeight, right of=worker1, node distance=2cm] (right1) {};
    \node [UWidth, workerHeight, right of=worker2, node distance=2cm] (right2) {};
    \node [UWidth, workerHeight, right of=worker3, node distance=2cm] (right3) {};

    \draw[fill=gray!12] (left1.north west) -- (left1.west |- passer1.south) -- 
          (right1.east |- passer1.south) -- (right1.north east) -- 
          (right1.north west) -- (right1.west |- passer1.north) -- 
          (left1.east |- passer1.north) -- (left1.north east) -- cycle;
    \draw[fill=gray!12] (left2.north west) -- (left2.west |- passer2.south) -- 
          (right2.east |- passer2.south) -- (right2.north east) -- 
          (right2.north west) -- (right2.west |- passer2.north) -- 
          (left2.east |- passer2.north) -- (left2.north east) -- cycle;
    \draw[fill=gray!12] (left3.north west) -- (left3.west |- passer3.south) -- 
          (right3.east |- passer3.south) -- (right3.north east) -- 
          (right3.north west) -- (right3.west |- passer3.north) -- 
          (left3.east |- passer3.north) -- (left3.north east) -- cycle;

    \node [listener, left of=left1, node distance=4cm] (listener) {Listener};
    \node [writer, right of=right1, node distance=3cm] (writer) {Writer};
    % Edges
    \path [line] (listener) -- (lleft1);
    \path [line] (lleft1) -- (left1);
    \path [line] ($ (lleft1) !.4! (left1) $) |- (left2);
    \path [line] ($ (lleft1) !.4! (left1) $) |- (left3);

    \path [line, dashed,transform canvas={shift={(0,-0.5)}}] (lleft1) -- (left1);
    \path [line, dashed,transform canvas={shift={(0,-0.5)}}] (lleft2) -- (left2);
    \path [line, dashed,transform canvas={shift={(0,-0.5)}}] (lleft3) -- (left3);

    \path [line] (left1) -- (worker1);
    \path [line] (left2) -- (worker2);
    \path [line] (left3) -- (worker3);

    \path [line] (worker1) -- (right1);
    \path [line] (worker2) -- (right2);
    \path [line] (worker3) -- (right3);
    \path [line] (right1) -- (writer);
    \path [draw] (right2) -| ($ (right1) !.3! (writer) $);
    \path [draw] (right3) -| ($ (right1) !.3! (writer) $);
    \node[xshift=0.5em, draw, densely dotted, minimum width=18em, inner sep=0.5em,fit=(right2) (right3) (lleft2) (lleft3) (passer3)] (balancer) {};
    \node [above of=balancer, node distance=4.8cm] {Simple Load Balancer};
\end{tikzpicture}
\caption[scale=1.0]{Simple server example showing usage of \texttt{StaticLoadBalancer}.}
\label{fig:exampleServer2}
\end{figure}

\subsection{Related work}
Similar work has been done by James Whitehead II in \cite{whitehead}.
However, his approach is rather different than the approach used 
here. He uses reader-writer pairs for communication between
two different components in the network. These act like channels, but
have to be constructed for each pair of communicating components
for all requests.
In order to construct these readers and writers a connection object is 
passed through the network.
I believe this indirect approach introduces unnecessary overhead and 
makes it more difficult to understand how the toolkit works.
More significantly it also restricts inter-component communication 
to a stream of bytes.

\subsection{Report overview}
In this Chapter I introduced the motivation of the project, its main
aim, the architecture of the proposed toolkit and I reviewed work done
on similar problems. In Chapter \ref{sec:background}
I present the necessary background for the project.

In Chapter \ref{sec:arch} I analyze the design of a few simple servers
in the suggested framework. Chapter \ref{sec:impl} describes the basic parts of 
my implementation of the proposed toolkit. Chapter \ref{sec:impl2} then builds
on this and introduces more advanced features of the implementation.

Afterwards in Chapter \ref{sec:examples} I present
implementations of servers described in Chapter \ref{sec:arch} in my framework.
In Chapter \ref{sec:shopping} I perform a case study of building
a part of a simple shopping server using my toolkit.

In chapter \ref{sec:test} I present results of performance tests of equivalent 
servers implemented in my framework, in pure go, using toolkit by James Whitehead II
\cite{whitehead} and using Apache framework. Finally in chapter 
\ref{sec:conclusion} I summarize the results and achievements of this 
project and present ideas for future work.