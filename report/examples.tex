\section{Examples}
\label{sec:examples}
\subsection{Hello world! server}
Figure \ref{fig:HelloWorldImpl} shows the Implementation of the Hello World!
server described in section \ref{sec:helloWorld}.
It's a direct translation of the diagram shown in Figure \ref{fig:helloWorld}.

The Listener corresponds to the call of mpserver.Listen function.
Hello world Component translates into an instance of ConstantComponent
and String Writer is the writer in the package with the same name.
We plug these together using two channels, start the components 
and then start the server itself.

\begin{figure}[h]
\centering
\begin{lstlisting}
package main

import(
    "log"
    "net/http"
    "mpserver"
)

func main() {
    in := mpserver.GetChan()
    out := mpserver.GetChan()

    mux := http.NewServeMux()
    mpserver.Listen(mux, "/hello", in)

    go mpserver.ConstantComponent("Hello world!")(in, out)
    go mpserver.StringWriter(out, nil)
    
    log.Println("Listening on port 3000...")
    http.ListenAndServe(":3000", mux)
}
\end{lstlisting}
\caption[scale=1.0]{Implementation of the Hello World! server.}
\label{fig:HelloWorldImpl}
\end{figure}

\newpage
\subsection{File server}
Figure \ref{fig:FileServerImpl} shows the implementation of a simple 
file server that compresses all '.go' files. It is almost a direct translation
of the diagram shown in figure \ref{fig:fileServer2} in section \ref{sec:fileServer}.
However, here the File Getter is composed of two parts, which are \texttt{PathMaker}
and \texttt{FileComponent}. The \texttt{PathMaker} firstly strips a given
prefix from the request path and then prepends a provided directory name to 
the result. The \texttt{FileComponent} tries to open a file at a path 
provided in the result field of the incoming value. Other modifications
are the addition of the \texttt{ErrorWriter} and the channel going to it from
the \texttt{Splitter}.
This path is used when a non existent file is requested be the client.
Otherwise '.go' files goes to the \texttt{GzipWriter} and all other files to 
the standard \texttt{FileWriter}.
The diagram that represents this server is shown in 
Figure \ref{fig:fileServer3}.

\begin{figure}[h]
\centering
\begin{tikzpicture}[node distance = 2cm, auto]
    % Nodes
    \node [listener] (listener) {Listener};
    \node [component, right of=listener, node distance=3cm] (maker) {Path Maker};
    \node [component, right of=maker, node distance=3cm] (comp) {File Component};
    \node [component, right of=comp, node distance=3cm] (splitter) {Splitter};
    \node [writer, right of=splitter, node distance=3cm] (fwriter) {File Writer};
    \node [writer, below of=fwriter] (gwriter) {Gzip Writer};
     \node [writer, below of=gwriter] (ewriter) {Error Writer};
    % Edges
    \path [line] (listener) -- (maker);
    \path [line] (maker) -- (comp);
    \path [line] (comp) -- (splitter);
    \path [line] (splitter) -- (fwriter);
    \path [line] (splitter) |- (gwriter);
    \path [line] (splitter) |- (ewriter);
\end{tikzpicture}
\caption[scale=1.0]{File server with compression and error handling.}
\label{fig:fileServer3}
\end{figure}

\begin{figure}
\begin{lstlisting}
package main
import(
    "log"
    "net/http"
    "mpserver"
    "strings"
)
// Test if the provided value contains an error in the Result
func isError(val mpserver.Value) bool {
    _, isErr := val.Result.(error)
    return isErr
}
// Test if the path of the requested file ends with .go
func isGoFile(val mpserver.Value) bool {
    return strings.HasSuffix(val.Request.URL.Path, ".go")
}

func main() {
    // Construct the channels
    in := mpserver.GetChan()
    toFileComp := mpserver.GetChan()
    toSplitter := mpserver.GetChan()
    compressed := mpserver.GetChan()
    errChan := mpserver.GetChan()
    splitterOut := mpserver.ToOutChans(
        []mpserver.ValueChan{errChan, compressed})
    uncompressed := mpserver.GetChan()

    // Start the file components
    go mpserver.PathMaker("files", "")(in, toFileComp)
    go mpserver.FileComponent(toFileComp, toSplitter)
    
    // Start the splitter
    go mpserver.Splitter(toSplitter, uncompressed, splitterOut, 
                         []mpserver.Condition{isError, isGoFile})

    // Start the writers
    go mpserver.GzipWriter(compressed, errChan)
    go mpserver.GenericWriter(uncompressed, errChan)
    go mpserver.ErrorWriter(errChan)

    // Start the server
    mux := http.NewServeMux()
    mpserver.Listen(mux, "/", in)
    log.Println("Listening on port 3000...")
    http.ListenAndServe(":3000", mux)
}
\end{lstlisting}
\caption[scale=1.0]{Implementation of a File Server that compresses go files.}
\label{fig:FileServerImpl}
\end{figure}

\newpage
\subsection{Proxy server}