\section{Examples}
\label{sec:examples}
This chapter presents the implementations of a Hello World! server,
a file server and a proxy server in the \texttt{mpserver} toolkit.

\subsection{Hello world! server}
Figure \ref{fig:HelloWorldImpl} shows the Implementation of the Hello World!
server described in section \ref{sec:helloWorld}. Figure \ref{fig:helloWorld2}
shows the same diagram as is shown in Figure \ref{fig:helloWorld}, but with
added channel names that are the same as the names used in the code.

\begin{figure}[h]
\centering
\begin{tikzpicture}[node distance = 2cm, auto]
    % Nodes
    \node [listener] (listener) {Listener};
    \node [component, right of=listener, text width=6em,, node distance=4cm] (hello) {Hello World Component};
    \node [writer, right of=hello, node distance=4cm] (writer) {String Writer};
    % Edges
    \path [line] (listener) edge node {in} (hello);
    \path [line] (hello) edge node {out} (writer);
\end{tikzpicture}
\caption[scale=1.0]{Hello world! server architecture with channel names.}
\label{fig:helloWorld2}
\end{figure}

The Listener corresponds to the call of \texttt{mpserver.Listen} function.
Hello world Component translates into an instance of \texttt{ConstantComponent}
and String Writer is the writer in the package with the same name.
We plug these together using two channels, start the component and writer 
and then start the server itself. This demonstrates that the diagram translates
directly into program code.

\begin{figure}[h]
\centering
\lstinputlisting{../examples/helloServer/helloServer.go}
\caption[scale=1.0]{Implementation of the Hello World! server.}
\label{fig:HelloWorldImpl}
\end{figure}

\newpage
\subsection{File server}
Figure \ref{fig:FileServerImpl} shows the implementation of a simple 
file server that compresses all `.go' files. It is almost a direct translation
of the diagram shown in Figure \ref{fig:fileServer2} in section \ref{sec:fileServer}.
However, here the File Getter is composed of two parts, which are \texttt{PathMaker}
and \texttt{FileComponent}.

The \texttt{PathMaker} firstly strips a given
prefix from the requested path and then prepends a provided directory name to 
the result. The \texttt{FileComponent} tries to open a file at a path 
provided in the result field of the input \texttt{job}. Other modifications
are the addition of the \texttt{ErrorWriter} and the channel going to it from
the \texttt{Router}.
This path is used when a non existent file is requested by the client.

Otherwise `.go' files are sent to the \texttt{GzipWriter} and all other files to 
the standard \texttt{FileWriter}.
The diagram that represents this server is shown in 
Figure \ref{fig:fileServer3}. Note that the diagram translates directly 
to program code again.

\begin{figure}[h]
\centering
\begin{tikzpicture}[node distance = 2cm, auto]
    % Nodes
    \node [listener] (listener) {Listener};
    \node [component, right of=listener, node distance=3cm] (maker) {Path Maker};
    \node [component, right of=maker, node distance=3cm] (comp) {File Component};
    \node [component, right of=comp, node distance=3cm] (splitter) {Router};
    \node [writer, right of=splitter, node distance=3cm] (fwriter) {File Writer};
    \node [writer, above of=fwriter] (gwriter) {Gzip Writer};
    \node [writer, above of=gwriter] (ewriter) {Error Writer};
    % Edges
    \path [line] (listener) -- (maker);
    \path [line] (maker) -- (comp);
    \path [line] (comp) -- (splitter);
    \path [line] (splitter) -- (fwriter);
    \path [line] ($(splitter.north) + (0.4, 0)$) |- node[condition, cross]{} (gwriter);
    \path [line] ($(splitter.north) + (-0.4, 0)$) |- node[condition, cross]{} (ewriter);
\end{tikzpicture}
\caption[scale=1.0]{File server with compression and error handling.}
\label{fig:fileServer3}
\end{figure}

\begin{figure}
\vspace{-1cm}
\lstinputlisting{../examples/fileServer/fileServer.go}
\caption[scale=1.0]{Implementation of a File Server that compresses go files.}
\label{fig:FileServerImpl}
\end{figure}

\newpage
\subsection{Proxy server}
A simple proxy server can be represented by a diagram shown in Figure
\ref{fig:proxyServer} below.
\begin{figure}[h]
\centering
\begin{tikzpicture}[node distance = 2cm, auto]
    % Nodes
    \node [listener] (listener) {Listener};
    \node [component, text width=6em, right of=listener, node distance=3cm] (proxy) {Proxy Component};
    \node [component, right of=proxy, node distance=3cm] (splitter) {Error Router};
    \node [writer, right of=splitter, node distance=3cm] (rwriter) {Response Writer};
    \node [writer, below of=rwriter] (ewriter) {Error Writer};
    
    \node [draw, densely dotted, inner sep=0.5em, fit=(splitter) (rwriter) (ewriter)] (container) {};
    \node [above of=container, node distance=2.3cm]{Writers};

    % Edges
    \path [line] (listener) -- (proxy);
    \path [line] (proxy) -- (splitter);
    \path [line] (splitter) -- (rwriter);
    \path [line] (splitter) |- (ewriter);

\end{tikzpicture}
\caption[scale=1.0]{Simple proxy server.}
\label{fig:proxyServer}
\end{figure}

Now to increase throughput of this server I propose wrapping the Proxy
Component in a Static Load Balancer. However, now the number of writers
will be the limiting factor, so I propose to wrap both writers and 
the router in Static Load Balancer for Writers. In order to be able 
to do this I create a joint writer that internally runs the router and
the writers as is indicated in the diagram of the Figure \ref{fig:proxyServer}.
The implementation is shown below in figure \ref{fig:ProxyWriters}.
\begin{figure}[h]
\lstinputlisting[firstline=12,lastline=18]{../examples/proxyServer/proxyServer.go}  
\caption[scale=1.0]{Implementation of the joint writer for the Proxy Server.}
\label{fig:ProxyWriters}
\end{figure}

To avoid doing the same request over and over again I propose
wrapping the Proxy Component again, but now in a Caching layer. The 
implementation of the server described so far as a single writer is shown
below in Figure \ref{fig:ProxyWriter}.
\begin{figure}[h]
\lstinputlisting[firstline=20,lastline=35]{../examples/proxyServer/proxyServer.go} 
\caption[scale=1.0]{Implementation of the writer representing the Proxy Server.}
\label{fig:ProxyWriter}
\end{figure}

\newpage
The server that uses the above writer can process approximately 10 requests
at once efficiently. That is if 10 requests arrive at approximately the same
time then they also finish at approximately the same if the server wasn't
doing any work when the requests arrived.
To increase this number dynamically based on current traffic I propose
to wrap the writer that represents the whole server in a Dynamic Load
Balancer. The code for running this server is shown in Figure 
\ref{fig:ProxyServerImpl}.
\begin{figure}[h]
\lstinputlisting[firstline=37,lastline=50]{../examples/proxyServer/proxyServer.go} 
\caption[scale=1.0]{Implementation of the the Proxy Server.}
\label{fig:ProxyServerImpl}
\end{figure}

This server will be able to process about 200 requests at once efficiently
at maximal capacity provided the computer on which it is run has enough resources.
The whole implementation is shown in Appendix \ref{sec:proxyServer.go}.

\subsection{Summary}
In this chapter I presented implementations of 3 progressively more 
complicated example servers. The throughput of the first two servers
can be increased in the same way as shown for the Proxy Server. The 
next section analyzes how to build part of a shopping server.


