\section{Implementation}
\label{sec:impl}
In this Chapter I present basic elements of my \texttt{mpserver} toolkit
and how to work with them. The most important are the primitive types:
\texttt{Value}, \texttt{Component} and \texttt{Writer}.

\subsection{Values}
\texttt{Value} is an interface that represents the type of messages that are
passed between the components of the network. It definition is shown
in Figure \ref{fig:Value} below.

\begin{figure}[h]
\centering
\begin{lstlisting}
type Value interface {
    //Public methods
    GetRequest() *http.Request
    GetResult() interface{}
    SetResult(interface{})
    SetResponseCode(int)
    SetResponseCodeIfUndef(int)
    SetHeader(string, string)

    // Private methods
    getResponseWriter() http.ResponseWriter
    getResponseCode() int
    writeHeader()
    write([]byte)
    close()
}
\end{lstlisting}
\caption[scale=1.0]{Declaration of the Value interface.}
\label{fig:Value}
\end{figure}

Note that in Go only types, functions, constants and variables whose name
starts with a capital letter are exported from a package. Hence, anyone
using the package can only use the first 6 methods of the \texttt{Value}
interface.

The interface is implemented by the struct type shown in Figure \ref{fig:value}
that is private to the package.
\newpage
\begin{figure}[h]
\centering
\begin{lstlisting}
type value struct {
    request *http.Request
    result interface{}

    responseCode int
    responseWriter http.ResponseWriter
    done chan<- bool
}
\end{lstlisting}
\caption[scale=1.0]{Declaration of a private type that implements the Value 
interface.}
\label{fig:value}
\end{figure}
Here:
\begin{itemize}
  \item \texttt{request} is a pointer to the HTTP request made by the 
        client implemented in the default HTTP library. It can be accessed
        using the \texttt{GetRequest} method. 

  \item \texttt{result} is the result computed so far (initially nil). This is
        of type \texttt{interface{}} which is implemented by any type.
        Hence, values of all types can be stored there. This field can be
        accessed and changed using the \texttt{GetResult} and \texttt{SetResult}
        methods respectively.
  
  \item \texttt{responseCode} is the HTTP response code that should be 
        returned to the client. The default value of this field is -1,
        which indicates that the responseCode has not been set yet. 
        It can be set using the \texttt{SetResponseCode} or 
        \texttt{SetResponseCodeIfUndef} methods, where the second method only
        makes a change in case the response code is undefined. The value
        of the response code can be retrieved using the \texttt{getResponseCode}
        method inside the package. This is intended to be used only by the 
        \texttt{Writer} type introduced later.

  \item \texttt{responseWriter} is an object that writes the response back 
        to the client. Again only \texttt{Writer}s use this field. They
        can retrieve it using \texttt{getResponseWriter} method. 
        Alternatively, they can use the \texttt{writeHeader} and
        \texttt{write} methods to access the field indirectly.
        These methods write the headers and 
        the body of the response to the client respectively.

  \item \texttt{done} is a signaling channel. Signal is sent on this channel,
		after the response is written at the end of the pipeline using the 
        \texttt{close} method. This allows the connection to the client to 
        be closed after the result is written.
\end{itemize}
Note that all methods related to writing the response back to the client
are private to the package. This is to ensure correct behavior, as otherwise
these methods could be accessed by users of the package in any part of the 
network, what could lead to exceptions.

\subsection{Components}
This section introduces the \texttt{Component} type and then shows how we can simply
create components. Afterwards it presents a way how multiple components can
be combined to form a linear pipeline. The sections ends with a description
of a component generator that implements simple load balancing among
multiple components.

\subsubsection{The Component type}
Component is a generic part of the pipeline with an input and output channel.
It processes the values provided on input channel and outputs the results
on the output channel. The type \texttt{Component} is defined as follows:
\begin{lstlisting}
type Component func (in <-chan Value, out chan<- Value)
\end{lstlisting}
Every component should satisfy the following conditions:
\begin{itemize}
    \item When run a component should eventually read all values that are
          written to its input channel.

    \item Every value that a component reads from its input channel
          should be written to its output channel after the value was processed
          and possibly modified. This ensures that every request gets a response.

    \item A component can terminate only when its input channel is closed 
          Before terminating, the component should close its output channel.
          This ensures termination of any components further down in the network.

    \item Component can only close its output channel and it can only do so
    	  after its input channel have been closed. It should never attempt
          to close the input channel as this will likely result in an exception.
\end{itemize}

\subsubsection{Making Components}
To simplify the construction of components, so that they adhere to 
the above specification the package provides a helper function that 
constructs a component using a function that maps one \texttt{Value} object 
into another \texttt{Value} object. 
That is the function has the following signature:
\begin{lstlisting}
type ComponentFunc func (val Value) Value
\end{lstlisting}
This represents the mapping of the input values to the output values. 
The implementation of the described component generator is shown in 
Figure \ref{fig:MakeComponent} below.
\begin{figure}[h]
\centering
\begin{lstlisting}
func MakeComponent(f ComponentFunc) Component {
    return func (in <-chan Value, out chan<- Value) {
        for val := range in {
            out <- f(val)
        }
        close(out)
    }
}
\end{lstlisting}
\caption[scale=1.0]{Declaration of MakeComponent function.}
\label{fig:MakeComponent}
\end{figure}

Here the for loop reads values from the input channel and assigns them
to the variable \texttt{val} while the input channel is open. Then the 
the function \texttt{f} is applied to this value and the result is outputted.
Afterwards, the component waits for another value. When the input channel
is closed the loop terminates. Afterwards, the component closes its output channel
and terminates.

\subsubsection{Constant Component}
Constant Component is a simple component generator that takes a value \texttt{c}
of any type and returns a component that writes \texttt{c} to the result field of 
all inputted values and then outputs them. It's definition using the 
\texttt{MakeComponent} generator is shown in Figure 
\ref{fig:ConstantComponent} below.
\begin{figure}[h]
\centering
\begin{lstlisting}
func ConstantComponent(c interface{}) Component {
    return MakeComponent(func (val Value) Value {
        val.SetResult(c); return val
    })
}
\end{lstlisting}
\caption[scale=1.0]{Declaration of constant component generator.}
\label{fig:ConstantComponent}
\end{figure}

\subsubsection{Linking components}
The library also provides a component generator \texttt{LinkComponents} that takes 
any number of components and returns a component that behaves as their 
linear combination. That is as a pipeline constructed from these components 
in the order in which they are provided. Its signature is shown below.
\begin{lstlisting}
func LinkComponents(components ...Component) Component
\end{lstlisting}
The subcomponents are linked and run when the combined component is run. 
The last of the provided components is run in the goroutine in which
the combined component was run.

\subsubsection{Simple Load Balancing}
To support running the same component with the same input and output 
channels multiple times the toolkit provides \texttt{SimpleLoadBalancer}.
Note that this is possible as channels in go are many-to-many.
\texttt{SimpleLoadBalancer} is a component generator that takes a worker 
component and number $n$ and returns a component that when run starts 
$n$ instances of the worker component. 

The case with $n=3$ in a simple pipeline is shown in Figure \ref{fig:slbEx} below. 
In order to support clean shutdown of the combined component, the load balancer uses 
wrappers (U shaped components in the diagram) around workers and a component that 
forwards incoming values to the workers. When input channel of the forwarding
component is closed it sends a signal to the component wrappers on shut 
down channels (dashed arrows in the diagram). The wrappers then close
the channels to the workers, which will force workers to terminate.
Finally the distributer closes the output channel of the whole component
and then the termination signal propagates further down the network.

\begin{figure}[h]
\centering
\begin{tikzpicture}[node distance = 1.5cm, auto,
    workerWidth/.style={text width=5em, text centered},
    workerHeight/.style={minimum height=4em},
    UWidth/.style={minimum width=0.6cm},
    UHeight/.style={minimum height=0.6cm}
    ]
    % Nodes
    \node [component] (worker1) {Worker};
    \node [workerWidth, UHeight, below of=worker1] (passer1) {};
    \node [workerWidth, UHeight, above of=worker1] (passer2) {};
    \node [component, above of=passer2] (worker2) {Worker};
    \node [component, below of=passer1] (worker3) {Worker};
    \node [workerWidth, UHeight, below of=worker3] (passer3) {};
    \node [above of=worker2, node distance=1.3cm, xshift=-0.8cm] {Simple Load Balancer};

    % Left
    \node [UWidth, workerHeight, left of=worker1, node distance=2cm] (left1) {};
    \node [UWidth, workerHeight, left of=worker2, node distance=2cm] (left2) {};
    \node [UWidth, workerHeight, left of=worker3, node distance=2cm] (left3) {};

    \node [UWidth, workerHeight, left of=left1, node distance=1.5cm] (lleft1) {};
    \node [UWidth, workerHeight, left of=left2, node distance=1.5cm] (lleft2) {};
    \node [UWidth, workerHeight, left of=left3, node distance=1.5cm] (lleft3) {};

    \draw (lleft2.north west) -- (lleft2.west |- passer3.south) --
          (lleft2.east |- passer3.south) -- (lleft2.north east) --
          cycle;

    % Right
    \node [UWidth, workerHeight, right of=worker1, node distance=2cm] (right1) {};
    \node [UWidth, workerHeight, right of=worker2, node distance=2cm] (right2) {};
    \node [UWidth, workerHeight, right of=worker3, node distance=2cm] (right3) {};

    \draw (left1.north west) -- (left1.west |- passer1.south) -- (right1.east |- 
          passer1.south) -- (right1.north east) -- (right1.north west) -- 
          (right1.west |- passer1.north) -- (left1.east |- passer1.north) -- 
          (left1.north east) -- cycle;
    \draw (left2.north west) -- (left2.west |- passer2.south) -- (right2.east |- 
          passer2.south) -- (right2.north east) -- (right2.north west) -- 
          (right2.west |- passer2.north) -- (left2.east |- passer2.north) -- 
          (left2.north east) -- cycle;
    \draw (left3.north west) -- (left3.west |- passer3.south) -- (right3.east |- 
          passer3.south) -- (right3.north east) -- (right3.north west) -- 
          (right3.west |- passer3.north) -- (left3.east |- passer3.north) -- 
          (left3.north east) -- cycle;

    \node [listener, left of=left1, node distance=4cm] (listener) {Listener};
    \node [writer, right of=right1, node distance=3cm] (writer) {Writer};
    % Edges
    \path [line] (listener) -- (lleft1);
    \path [line] (lleft1) -- (left1);
    \path [line] ($ (lleft1) !.4! (left1) $) |- (left2);
    \path [line] ($ (lleft1) !.4! (left1) $) |- (left3);

    \path [line, dashed,transform canvas={shift={(0,-0.5)}}] (lleft1) -- (left1);
    \path [line, dashed,transform canvas={shift={(0,-0.5)}}] (lleft2) -- (left2);
    \path [line, dashed,transform canvas={shift={(0,-0.5)}}] (lleft3) -- (left3);

    \path [line] (left1) -- (worker1);
    \path [line] (left2) -- (worker2);
    \path [line] (left3) -- (worker3);

    \path [line] (worker1) -- (right1);
    \path [line] (worker2) -- (right2);
    \path [line] (worker3) -- (right3);
    \path [line] (right1) -- (writer);
    \path [draw] (right2) -| ($ (right1) !.3! (writer) $);
    \path [draw] (right3) -| ($ (right1) !.3! (writer) $);
    \node[xshift=0.5em, draw, densely dotted, minimum width=18em, inner sep=0.5em,fit=(right2) (right3) (lleft2) (lleft3) (passer3)] {};
\end{tikzpicture}
\caption[scale=1.0]{Simple server example showing usage of \texttt{SimpleLoadBalancer}.}
\label{fig:slbEx}
\end{figure}

This combined component allows multiple requests to be processed by the worker
components at the same time and this in turn allows faster requests to 
overtake slower requests in the pipeline. The worker component can represent
a simple computation done locally or they can connect to other servers
and perform expensive computation there. More advanced version of load 
balancing component is introduced in the next Chapter.

\newpage
\subsection{Writers}
Here, I introduce the \texttt{Writer} type, which acts as the end of the 
pipeline. Then I present a way how custom writers can be implemented.
\subsubsection{Writer type}
Writer is the end of the pipeline which writes results to the client.
The type \texttt{Writer} is defined as follows:
\begin{lstlisting}
type Writer func (in <-chan Value)
\end{lstlisting}
It is a function with only an input channel.
Every writer satisfies the following specification.
\begin{itemize}
    \item When run a writer eventually reads all values that are written 
          to its input channel.

	\item For every value that is input any writer behaves as follows. 
          It firstly checks if the result field contains the expected type. 
          If that is the case, it firstly writes the headers, then
          writes the body and finally closes the connection for the value
          using the \texttt{writeHeader}, \texttt{write} and \texttt{close}
          methods of the value respectively. If the value in the result 
          field has an incorrect type the writer writes an appropriate 
          error message to the client and closes the connection for the value. 

	\item The writer terminates when its input channel is closed.
\end{itemize}
The package provides writers for writing string, json, file, error and many
other types of responses. There is also a writer generator for static
load balancing similar to the one for components. Note that components
combined with writers appear like a single writer. Hence, we can use the
load balancers for writers to manage a more complex part the network.

\subsubsection{Making Writers}
As noted before methods for writing responses from \texttt{Value}s 
back to the client can not be used outside of the package. To allow
for construction of custom writers the package provides writer generator
component similar to the \texttt{MakeComponent} function. It uses 
functions with the following signature to generate the writers.
\begin{lstlisting}
type WriterFunc func (val Value) ([]byte, error)
\end{lstlisting}
These functions should set headers and convert the provided value into 
an array of bytes or return an error in case they are provided with a 
wrong input. The implementation of the writer generator itself is shown
in Figure \ref{fig:MakeWriter} below.
\begin{figure}[h]
\centering
\begin{lstlisting}
func MakeWriter(writerFunc WriterFunc) Writer {
    return func (in <-chan Value) {
        for val := range in {
            resp, err := writerFunc(val)

            if (err == nil) {
                // Write the response
                val.SetResponseCodeIfUndef(http.StatusOK)
                val.writeHeader()
                val.write(resp)
                val.close()
            } else {
                WriteError(val, err)
            }
        }       
    }
}
\end{lstlisting}
\caption[scale=1.0]{Implementation of writer generator.}
\label{fig:MakeWriter}
\end{figure}

The generated writer runs the provided \texttt{writerFunc} for every value 
that is inputted. If the function returns an error it writes the error 
back to the client using the \texttt{WriteError} function. Otherwise, it 
writes the headers and the body of the response to the client and then 
closes the connection. (The \texttt{WriteError} does this for the error
results).

\subsection{Splitter and Collector}
In this section I introduce a way to split a stream of incoming values into
multiple streams of values that go into different parts of the network
using very flexible rules for splitting.

\subsubsection{Conditions}
Condition type is defined as follows:
\begin{lstlisting}
type Condition func (val Value) bool
\end{lstlisting}
It is a just function that takes a value and returns a boolean.

\subsubsection{Splitter}
Conditions are used in Splitter, which is a component with the 
following signature:
\begin{lstlisting}
func Splitter(in <-chan Value, defOut chan<- Value, 
			  outs []chan<- Value, conds []Condition)
\end{lstlisting}
It takes an input channel, default output channel, array of output channels and array of conditions.
The number of output channels and the number of conditions should be the same.
For every input value the Splitter evaluates the conditions from first to last.
When a condition returns true the value is written to a corresponding output channel 
and the processing of the current value terminates. If all conditions return false
then the value is written to the default output channel.

\subsubsection{Error Splitter}
The package also implements a special splitter that redirects error messages
to a special error channel and sends all other messages to the default output 
channel. This is used to avoid sending error values to 
any other type of writer than the error writer. If we allowed this, then
all errors will only show that a particular writer got a wrong input.
The signature of the \texttt{ErrorSplitter} function is shown below.
\begin{lstlisting}
func ErrorSplitter(in <-chan Value, out chan<- Value, 
                                    errChan chan<- Value)
\end{lstlisting}

\subsubsection{Collector}
Collector is a component whose behavior is the opposite of splitter. 
It gets multiple input channels
and forwards their outputs to a single output channel. Now as channels
in go are many-to-many we could just use a single channel to which all
multiple components output their values. 

However, when we shutdown
one of the components, this component will close the output channel
what will cause panic next time one of the other components tries to write
to it. Now, \texttt{Collector} only closes its output channel, when
all of its input channels have been closed, thus avoiding the above
problem. It is acceptable to just use a single channel instead of
the collector component, if we don't intend to shut down the network.
The type signature of the \texttt{Collector} is show below.
\begin{lstlisting}
func Collector(ins []<-chan Value, out chan<- Value)
\end{lstlisting}


\subsection{Summary}
In this Chapter I introduced the basic elements of the library. This 
is enough to implement some simple serves. The next section introduces
more advanced components that allow more complex behaviors.

