\section{Background}
\label{sec:background}
In this Chapter I explain the basics of message passing concurrency and 
the basics of the go programming language, which was used to implement 
the proposed toolkit.

\subsection{Introduction to Message passing Concurrency}
This section first introduces channels and then presents the Message
Passing Concurrency Paradigm.

\subsubsection{Channels}
Channel is a FIFO queue of pending messages. It can be accessed by two
primitive operations: send and receive. To start communication
a process sends a value to the channel. Another process acquires the message
by receiving from the channel. Sending a message can be asynchronous (nonblocking)
or synchronous (blocking). \cite[293]{book:foundations}

\subsubsection{Message Passing Concurrency}
Message Passing Concurrency is a style of Concurrent Programming, in which
processes communicate using only channels. That is channel communication is 
the only way any two processes can communicate and synchronize. 
Hence this style does not 
suffer any problems that arise from using shared memory for communication 
and synchronization by multiple concurrent processes.

\subsection{Introduction to go}
``Go is a statically-typed programming language. Programs are compiled 
to native code and are linked with a small runtime environment that performs 
automatic memory management and scheduling for lightweight processes called 
goroutines.'' \cite[2]{whitehead} Its main advantage is 
that it has concurrent constructs built natively into it,
and this is why it was chosen as implementation language 
for this project.
In this section I introduce the main concurrency constructs that are
present in the language, its type system and exception handling. Other 
features of the language are explained as they are used.

\subsubsection{Goroutines}
``A goroutine is a cross between a lightweight thread and a coroutine
managed by the Go runtime.'' \cite[2]{whitehead} Goroutines are multiplexed
on a set of threads, when a goroutine blocks another goroutine is scheduled
on the same thread. 

What makes goroutines powerful is that they are very cheap to create and 
it is also cheap to switch between them. The only significant cost when 
creating a goroutine is the memory allocation for the stack which is 
usually only a few kilobytes \cite{FAQ}. ``The CPU overhead averages about 
three cheap instructions per function call.''\cite{FAQ} These costs are much 
smaller than using operating system threads, \textit{hence it is feasible to use 
thousands of goroutines}.

To create a new goroutine we only need to prefix a function call with
the \texttt{go} keyword. The arguments will be evaluated in the current
goroutine, but unlike with regular function call this goroutine won't
wait for the function to return. The called 
function will be evaluated in a new goroutine and that goroutine will 
terminate when the function terminates \cite{GoDocumentation}.

\subsubsection{Channels}
Channels in go are many-to-many. That is multiple goroutines can
read from and write to a single channel. Channels are statically typed and
dynamically allocated using \texttt{make} function. This function takes 
two arguments: the type of the channel and the size of its buffer.
A channel with buffer size 0 implements a synchronous channel, whereas 
channels with buffers implement asynchronous channels, as reads
are only blocked when the buffer is empty and writes are only blocked
when the buffer is full.

By default the channel type exposes both reading and writing end of the
channel. Hence, we can view it as a bidirectional channel. 
However, we can restrict the access to only one of its ends using the following:
\begin{lstlisting}
channel := make(chan int, 10)
in := (<-chan int)(channel)
out := (chan<- int)(channel)
\end{lstlisting}
Here, we define 3 variables. \texttt{channel} variable refers to the 
channel itself, \texttt{in} refers to the reading end of the channel
and \texttt{out} to the writing end of the channel. We used casting
from the channel to access only one of its end. The \texttt{:=} is a
short form of variable declaration with assignment and an implicit type.

\subsubsection{Objects methods and Interfaces}
Go doesn't have classes and inheritance as Object Oriented Languages 
such as Java do. It instead allows programmers to define named types
and methods on these types. This is illustrated in the example shown in 
Figure \ref{fig:counterObj}, which defines a simple counter object.
Note that the operations are defined on a pointer to a \texttt{Counter}
object. This is because all function parameters are passed by value, so
if we defined the update method on Counter object, calling it wouldn't
change the value of the Counter.

\begin{figure}[h]
\centering
\begin{lstlisting}
type Counter struct {
    count int
}
func (c *Counter) update(k int) {c.count += k}
func (c *Counter) getCount() int {return c.count}
\end{lstlisting}
\caption[scale=1.0]{Definition of a simple \texttt{Counter} object.}
\label{fig:counterObj}
\end{figure}

To specify behavior of types go provides interfaces. An interface
specifies methods that a type must implement in order to satisfy it.
``Interfaces are satisfied implicitly. That is a type implements 
an interface by implementing its methods.'' \cite{tour}
\texttt{CounterInterface}, defined in Figure \ref{fig:counterInter} below,
defines methods that any counter object should provide. This interface 
is implemented by a pointer to a \texttt{Counter} object defined above.

\begin{figure}[h]
\centering
\begin{lstlisting}
type CounterInterface interface {
    update(int)
    getCount() int
}
\end{lstlisting}
\caption[scale=1.0]{Definition of a \texttt{CounterInterface}.}
\label{fig:counterInter}
\end{figure}

\subsubsection{Panic and Recover}
Go doesn't support standard form of exceptions as are found in Java or
C++. It instead implements \texttt{panic} and \texttt{recover} methods, 
which are very similar to Java's \texttt{throw} and \texttt{catch}.
\texttt{panic} can be called explicitly or it can be a result of 
a run time error such as writing to a closed channel. We say that 
a goroutine panics when \texttt{panic} is called.

% Maybe use the stuff below, but is not necessary
% When \texttt{panic} is called inside a function \texttt{F}, the execution
% of \texttt{F} is stopped and any functions that have been deferred by
% \texttt{F} are executed. ``Next, any deferred functions run by \texttt{F}'s caller 
% are run, and so on up to any deferred by the top-level function in the executing 
% goroutine. Then the program is terminated and error condition 
% is reported. This termination sequence is called panicking.'' \cite{goSpec}
% A call function to function \texttt{H} is deferred by prefixing it with a 
% \texttt{defer} keyword. This function is called when the function that 
% called it terminates.

% Panic sequence can be stopped in case one of the deferred functions
% calls the \texttt{recover} method and then returns normally. The value
% returned by the \texttt{recover} method is the same value that was passed
% to the call of panic \texttt{panic} or \texttt{nil} if the goroutine
% is not panicking. \cite{goSpec}

\subsection{Summary}
In this chapter I introduced the basics of the message passing concurrency
paradigm and the go programming language. In the next chapter I 
analyze conceptual design of a few example servers in the proposed
architecture.



