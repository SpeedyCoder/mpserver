\section{Building a shopping server}
\label{sec:shopping}
In this Chapter I show how to build a web-server using my toolkit that
maintains a shopping cart for a simple shopping web-site. The whole 
code for this example is shown in Appendix A.

\subsection{Shopping cart}
Shopping cart is represented by the struct type showed in Figure 
\ref{fig:shoppingCart} below.
\begin{figure}[h]
\begin{lstlisting}
type ShoppingCart struct {
    items map[string]int
    bought bool
}
\end{lstlisting}
\caption[scale=1.0]{Type representing a shopping cart.}
\label{fig:shoppingCart}
\end{figure}

Here, \texttt{items} is 
a mapping that represents the contents of the cart. It stores the names
of the items and their count. \texttt{bought} indicates whether the items
were purchased. In a realistic implementation one might choose to use 
a list of ids rather than a map, but using a map keeps things simple
for the purposes of this example.

\subsection{Actions}
To represent actions that a user can take I used the interface show in 
Figure \ref{fig:action}.
\begin{figure}[h]
\begin{lstlisting}
type Action interface {
    performAction(s ShoppingCart) (mpserver.State, error)
}
\end{lstlisting}
\caption[scale=1.0]{Type representing an action that can be taken on 
a shopping cart.}
\label{fig:action}
\end{figure}
That is any action implements a \texttt{performAction} method, which takes
a \texttt{ShoppingCart} object and tries to perform its operation on it. If
it succeed then it returns the updated object and \texttt{nil} pointer for the
error. If it does not succeed then the method returns \texttt{nil} for the
object and the corresponding error.

The \texttt{Action} interface is implemented by the following 3 types
that represent adding items to the shopping cart, removing an item from it
and buying the contents of the shopping cart.
\begin{figure}[h]
\begin{lstlisting}
type AddAction struct {
    items []string
}
type RemoveAction struct {
    item string
}
type BuyAction struct {}
\end{lstlisting}
\caption[scale=1.0]{Types representing simple actions.}
\label{fig:actions}
\end{figure}

Now Figure \ref{fig:addAction} shows the implementation of the \texttt{performAction}
method for the \texttt{addAction}. This action creates the map object if necessary and
then adds contents of the items array of the \texttt{addAction} object to the map
object in the \texttt{shoppingCart}.
\begin{figure}[h]
\begin{lstlisting}
func (a AddAction) performAction(s ShoppingCart) (State, error) {
    if (s.items == nil) {
        s.items = make(map[string]int)
    }
    for _, item := range a.items {
        s.items[item] += 1
    }
    return s, nil
}
\end{lstlisting}
\caption[scale=1.0]{\texttt{performAction} implementation by \texttt{addAction}.}
\label{fig:addAction}
\end{figure}

\subsection{Implementing State interface}
Now we want the \texttt{ShoppingCart} type to implement the \texttt{State} 
interface defined in section
\ref{sec:state}, so that it can be used by the session manager component.
This component will maintain one \texttt{ShoppingCart} object for each user.
Figure \ref{fig:shoppingCartImpl} below shows the implementation of the 
methods of the \texttt{State} interface.
\begin{figure}[h]
\begin{lstlisting}
func (s ShoppingCart) Next(val Value) (State, error) {
    a, ok := val.GetResult().(Action)
    if (!ok) {
        return nil, errors.New("Action not provided")
    }

    return a.performAction(s)
}
func (s ShoppingCart) Result() interface{} {
    return s.items
}
func (s ShoppingCart) Terminal() bool {
    return s.bought
}
\end{lstlisting}
\caption[scale=1.0]{Implementation of the methods for \texttt{ShoppingCart}.}
\label{fig:shoppingCartImpl}
\end{figure}

These are all very simple methods. The \texttt{Next} method calls 
the \texttt{performAction} method on the provided action to generate 
the next state.

\subsection{Making actions}
Now we need a way to create actions from requests. We create three different
components to do that using the \texttt{MakeComponent} function. Below
is the implementation of the function used to make the \texttt{addAction}.
\begin{figure}[h]
\begin{lstlisting}
func addActionMakerFunc(val Value) Value {
    q := val.GetRequest().URL.Query()
    items, ok := q["item"]
    if (ok && len(items) > 0) {
        items = strings.Split(items[0], ",")
    }
    val.SetResult(AddAction{items})
    
    return val
}
var addActionMaker = MakeComponent(addActionMakerFunc) 
\end{lstlisting}
\caption[scale=1.0]{Implementation of the \texttt{addActionMaker}.}
\label{fig:addActionMaker}
\end{figure}

\subsection{Plugging it together}
We use different URL paths to perform different actions. We route the 
requests with paths starting with \texttt{/add} to the \texttt{addActionMaker}
and similarly for the other two actions. The output of the action maker
will be connected to the session management component, that is wrapped in
\texttt{ErrorPasser}, so that we don't process wrong requests.
Finally the output from the session manager goes to the \texttt{ErrorSplitter}
which sends all non-error values to to \texttt{JsonWriter} and error
values to \texttt{ErrorWriter}. The overall design is show in diagram 
of Figure \ref{fig:shoppingDesign}. The \texttt{ErrorPasser} is omitted 
for clarity. The code for running the server is shown in figure 
\ref{fig:shoppingCode}.

\begin{figure}[h]
\centering
\begin{tikzpicture}[node distance = 2cm, auto]
    % Nodes
    \node [listener] (listener1) {/add};
    \node [listener, below of=listener1] (listener2) {/remove};
    \node [listener, below of=listener2] (listener3) {/buy};
    \node [component, right of=listener1, node distance=3cm] (maker1) {Add Action Maker};
    \node [component, right of=listener2, node distance=3cm] (maker2) {Remove Action Maker};
    \node [component, right of=listener3, node distance=3cm] (maker3) {Buy Action Maker};
    \node [component, right of=maker2, node distance=3cm] (manager) {Session Manager};
    \node [component, right of=manager, node distance=3cm] (splitter) {Error Splitter};
    \node [writer, right of=splitter, node distance=3cm, yshift=1cm] (writer1) {JSON Writer};
    \node [writer, below of=writer1] (writer2) {Error Writer};

    % Edges
    \path [line] (listener1) -- (maker1);
    \path [line] (listener2) -- (maker2);
    \path [line] (listener3) -- (maker3);
    \path [draw] (maker1) -| ($ (maker2) !.5! (manager) $);
    \path [line] (maker2) -- (manager);
    \path [draw] (maker3) -| ($ (maker2) !.5! (manager) $);
    \path [line] (manager) -- (splitter);
    \path [line] (splitter) |- (writer1);
    \path [line] (splitter) |- (writer2);


\end{tikzpicture}
\caption[scale=1.0]{Shopping server architecture.}
\label{fig:shoppingDesign}
\end{figure}

\newpage
\begin{figure}[h]
\begin{lstlisting}
func main() {
    // Create channels
    toAddActionMaker := make(mpserver.ValueChan)
    toRemoveActionMaker := make(mpserver.ValueChan)
    toBuyActionMaker := mpserver.GetChan()
    in := mpserver.GetChan()
    out := mpserver.GetChan()
    toWriter := mpserver.GetChan()
    errChan := mpserver.GetChan()

    // Action makers
    go addActionMaker(toAddActionMaker, in)
    go removeActionMaker(toRemoveActionMaker, in)
    go buyActionMaker(toBuyActionMaker, in)

    // Session Manager
    initial := ShoppingCart{nil, false}
    store := mpserver.NewMemStore()
    sComp := mpserver.SessionManagementComponent(
        store, initial, time.Second*60*5, true)
    sComp = mpserver.ErrorPasser(sComp)
    go sComp(in, out)

    // Writers
    go mpserver.ErrorSplitter(out, toWriter, errChan)
    go mpserver.JsonWriter(toWriter)
    go mpserver.ErrorWriter(errChan)

    mux := http.NewServeMux()
    mpserver.Listen(mux, "/add", toAddActionMaker)
    mpserver.Listen(mux, "/remove", toRemoveActionMaker)
    mpserver.Listen(mux, "/buy", toBuyActionMaker)
    log.Println("Listening on port 3000...")
    http.ListenAndServe(":3000", mux)
}
\end{lstlisting}
\caption[scale=1.0]{Implementation of the Shopping Server.}
\label{fig:shoppingCode}
\end{figure}