\section{Building a shopping server}
\label{sec:shopping}
In this Chapter I show how to build a web server that
maintains a shopping cart for a simple shopping web site. The whole 
code for this example is shown in Appendix~\ref{sec:shopping.go}.

\subsection{Shopping cart}
Shopping cart is represented by the struct type showed in 
Figure~\ref{fig:shoppingCart} below.
\begin{figure}[h]
\begin{lstlisting}
type ShoppingCart struct {
    items map[string]int
    bought bool
}
\end{lstlisting}
\caption[scale=1.0]{Type representing a shopping cart.}
\label{fig:shoppingCart}
\end{figure}

Here, \texttt{items} is 
a mapping that represents the contents of the cart. It stores the names
of the items and their count. \texttt{bought} indicates whether the items
were purchased. In a real life scenario one might choose to use 
a list of Ids rather than a map, but using a map keeps things simple
for the purposes of this example.

\subsection{Actions}
To represent actions that a user can take I used the interface shown in 
Figure~\ref{fig:action}.
\begin{figure}[h]
\begin{lstlisting}
type Action interface {
    performAction(s ShoppingCart) (mpserver.State, error)
}
\end{lstlisting}
\caption[scale=1.0]{Type representing an action that can be taken on 
a shopping cart.}
\label{fig:action}
\end{figure}

That is any action must implement the \texttt{performAction} method, which takes
a \texttt{ShoppingCart} object and tries to perform its operation on it. If
it succeeds then it returns the updated object and \texttt{nil} pointer for the
error. If it does not succeed then the method returns \texttt{nil} for the
object and a corresponding error.

The \texttt{Action} interface is implemented by the following three types
that represent adding items to the shopping cart, removing an item from it
and buying the contents of the shopping cart.
\begin{figure}[h]
\begin{lstlisting}
type AddAction struct {
    items []string
}
type RemoveAction struct {
    item string
}
type BuyAction struct {}
\end{lstlisting}
\caption[scale=1.0]{Types representing simple actions.}
\label{fig:actions}
\end{figure}

Now Figure~\ref{fig:addAction} shows the implementation of the \texttt{performAction}
method for the \texttt{addAction}. This action creates the map object if 
necessary and then adds the contents of the items array in the \texttt{AddAction}
object to the items map in the \texttt{shoppingCart}.
\begin{figure}[h]
\begin{lstlisting}
func (a AddAction) performAction(s ShoppingCart) (State, error) {
    if (s.items == nil) {
        s.items = make(map[string]int)
    }
    for _, item := range a.items {
        s.items[item] += 1
    }
    return s, nil
}
\end{lstlisting}
\caption[scale=1.0]{\texttt{performAction} implementation for \texttt{AddAction}.}
\label{fig:addAction}
\end{figure}

\subsection{Implementing State interface}
Now we want the \texttt{ShoppingCart} type to implement the \texttt{State} 
interface defined in section~\ref{sec:state}, 
so that it can be used by the session manager component.
This component will maintain one \texttt{ShoppingCart} object for each user.
Figure~\ref{fig:shoppingCartImpl} below shows the implementation of the 
methods of the \texttt{State} interface for the \texttt{ShoppingCart} type.
\begin{figure}[h]
\begin{lstlisting}
func (s ShoppingCart) Next(job mpserver.Job) (
                                         mpserver.State, error) {
    a, ok := job.GetResult().(Action)
    if (!ok) {
        return nil, errors.New("No action provided")
    }
    return a.performAction(s)
}
func (s ShoppingCart) Result() interface{} {
    return s.items
}
func (s ShoppingCart) Terminal() bool {
    return s.bought
}
\end{lstlisting}
\caption[scale=1.0]{Implementation of the methods for \texttt{ShoppingCart}.}
\label{fig:shoppingCartImpl}
\end{figure}
These are all very simple methods. The \texttt{Next} method calls 
the \texttt{performAction} method on the provided action to generate 
the next state.

\subsection{Making actions}
Now we need a way to create actions from requests. We create three different
components to do that using the \texttt{MakeComponent} function. Below
is the implementation of the component that is used to generate 
the \texttt{addActions}.
\begin{figure}[h]
\begin{lstlisting}
func addActionMakerFunc(job mpserver.Job) {
    q := job.GetRequest().URL.Query()
    items, ok := q["item"]
    if (ok && len(items) > 0) {
        items = strings.Split(items[0], ",")
    }
    job.SetResult(AddAction{items})
}
var addActionMaker = MakeComponent(addActionMakerFunc) 
\end{lstlisting}
\caption[scale=1.0]{Implementation of the \texttt{addActionMaker}.}
\label{fig:addActionMaker}
\end{figure}

\subsection{Plugging it together}
We use different URL paths to perform different actions. We route the 
requests with paths starting with \texttt{/add} to the \texttt{addActionMaker}
and similarly for the other two actions. The output of each action maker
will be connected to the \texttt{SessionManager}, which is wrapped in
\texttt{ErrorPasser} so that we don't process invalid requests.
Finally, the output from the session manager will go to the \texttt{ErrorRouter}
which sends all \texttt{jobs} with non-error results to the 
\texttt{JsonWriter} and \texttt{jobs} with error results to the 
\texttt{ErrorWriter}. 

We could share the \texttt{SessionManager}, the router and the writers for all three actions. 
However, this will make it difficult to load balance individual parts
of the network. Hence, each action maker should be connected to their
own session manager and this to its own router and writers. For the 
server to function properly the session managers have to share a 
\texttt{Storage} object. The overall design of the server is shown in 
Figure~\ref{fig:shoppingDesign}. The \texttt{ErrorPasser} is omitted for clarity. 
\begin{figure}[h]
\centering
\begin{tikzpicture}[node distance = 2cm, auto]
    % Nodes
    \node [listener] (listener) {/add};
    \node [component, right of=listener, node distance=3cm] (maker) {Add Action Maker};
    \node [component, right of=maker, node distance=3cm] (manager) {Session Manager};
    \node [component, right of=manager, node distance=3cm] (splitter) {Error Router};
    \node [writer, right of=splitter, node distance=3cm, yshift=1cm] (writer) {JSON Writer};
    \node [writer, below of=writer] (ewriter) {Error Writer};

    \node [listener, below of=listener, node distance=4cm] (listener2) {/remove};
    \node [component, right of=listener2, node distance=3cm] (maker2) {Remove Action Maker};
    \node [component, right of=maker2, node distance=3cm] (manager2) {Session Manager};
    \node [component, right of=manager2, node distance=3cm] (splitter2) {Error Router};
    \node [writer, right of=splitter2, node distance=3cm, yshift=1cm] (writer2) {JSON Writer};
    \node [writer, below of=writer2] (ewriter2) {Error Writer};

    \node [listener, below of=listener2, node distance=4cm] (listener3) {/buy};
    \node [component, right of=listener3, node distance=3cm] (maker3) {Buy Action Maker};
    \node [component, right of=maker3, node distance=3cm] (manager3) {Session Manager};
    \node [component, right of=manager3, node distance=3cm] (splitter3) {Error Router};
    \node [writer, right of=splitter3, node distance=3cm, yshift=1cm] (writer3) {JSON Writer};
    \node [writer, below of=writer3] (ewriter3) {Error Writer};

    \node [draw, ellipse, text centered, text width=5em, above of=manager, 
           xshift=1.5cm, fill=gray!12, node distance=1.5cm] (storage) {Storage};
    
    % Edges
    \path [draw, line width=0.46mm, color=gray] ($(manager.east) + (0,0.4)$) -| (storage);
    \path [draw, line width=0.46mm, color=gray] ($(manager2.east) + (0,0.4)$) -| (storage);
    \path [draw, line width=0.46mm, color=gray] ($(manager3.east) + (0,0.4)$) -| (storage);

    \path [line] (listener) -- (maker);
    \path [line] (maker) -- (manager);
    \path [line] (manager) -- (splitter);
    \path [line] (splitter) |- (writer);
    \path [line] (splitter) |- (ewriter);

    \path [line] (listener2) -- (maker2);
    \path [line] (maker2) -- (manager2);
    \path [line] (manager2) -- (splitter2);
    \path [line] (splitter2) |- (writer2);
    \path [line] (splitter2) |- (ewriter2);

    \path [line] (listener3) -- (maker3);
    \path [line] (maker3) -- (manager3);
    \path [line] (manager3) -- (splitter3);
    \path [line] (splitter3) |- (writer3);
    \path [line] (splitter3) |- (ewriter3);

\end{tikzpicture}
\caption[scale=1.0]{Shopping server architecture.}
\label{fig:shoppingDesign}
\end{figure}

We will make each part of the network into a separate writer as in
the Proxy Server example to be able to use the \texttt{DynamicLoadBalancer}.
The code for making this writer is shown in Figure~\ref{fig:shoppingWriter}.
Here, we provide the action maker and shared storage object for the session manager 
as arguments of the function.

\begin{figure}[h]
\lstinputlisting[firstline=12,lastline=35]{../examples/shoppingServer/shoppingServer.go} 
\caption[scale=1.0]{Implementation of the writer that represents the whole
pipeline for one action.}
\label{fig:shoppingWriter}
\end{figure}

The benefit of the chosen architecture is that resources for the add, remove
and buy actions are allocated independently. That is, if there is lot of 
people adding items to their shopping cart, there can be more instances
of the combined writer for the add action and fewer instances for the
other two actions. The code for running the whole server is shown
in Figure~\ref{fig:ShoppingCode}.

\begin{figure}
\lstinputlisting[firstline=37,lastline=72]{../examples/shoppingServer/shoppingServer.go} 
\caption[scale=1.0]{Implementation of the Shopping Server.}
\label{fig:ShoppingCode}
\end{figure}

\subsection{Summary}
In this chapter I showed how to use the \texttt{mpserver} toolkit
to build a more complex web server. I also presented how resources can
be allocated independently for different tasks. The next chapter 
presents the results of performance testing of the toolkit.


