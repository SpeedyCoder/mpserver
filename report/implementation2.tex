\section{Advanced component generators}
\label{sec:impl2}
In this section I introduce more advanced component generators for caching,
session management, load balancing, network communication and error handling.

\subsection{Storage}
Storage is an interface representing thread safe mapping from string keys
to values of any type. Its definition is shown below.
\begin{lstlisting}
type Storage interface {
    Get(string) (interface{}, bool)
    Set(string, interface{})
    Remove(string)
    CompareAndRemove(string, interface{}) bool
    Keys() []string
}
\end{lstlisting}
The package provides a default implementation of this interface that
stores the data in memory using (with minor modifications) sharded concurrent map implementation
by Or Hiltch\footnote{Code available at \url{https://github.com/orcaman/concurrent-map}}.
If a developer wants to store the values in the 
database instead, they can provide their own implementation of the store interface
and use any component that uses the storage the same way as before.
This makes the components that use the storage highly configurable.


\subsection{States and Session Management Component}
\subsubsection{States}
\label{sec:state}
State is an interface defined as follows:
\begin{figure}[h]
\centering
\begin{lstlisting}
type State interface {
    Next(val Value) (State, error)
    Terminal() bool
    Result() Any
}
\end{lstlisting}
\caption[scale=1.0]{Declaration of the State interface.}
\label{fig:State}
\end{figure}

Here the expected behavior of the member functions is as follows:
\begin{itemize}
	\item The \texttt{Next} function returns the next state when provided with a value.
	\item The \texttt{Terminal} function indicates whether the current state is a terminal state.
	\item The \texttt{Result} function returns the result that possibly after further 
		  processing should be returned to the client.
\end{itemize}

\subsubsection{Session Management Component}
\texttt{SessionManagementComponent} is a following component generator.
\begin{lstlisting}
func SessionManagementComponent(store Storage, initial State, 
                seshExp time.Duration, useCleaner bool) Component
\end{lstlisting}

It arguments are a storage object to be used, an initial state for each session,
a session expiration time and a boolean indicating whether expired
entries should be automatically removed from the map.
The returned component behaves as a session manager. 
It behaves as follows:

The component stores a mapping from session ids to current state of the session.

Every new request gets assigned a unique session id. Then a mapping from the generated 
id to the current state, which is generated from the initial state using the provided value,
is stored. The result of the call to the \texttt{Result} function on the next state with the provided 
value is then sent down the pipeline.

When a request with set `Session-Id` header comes in, the component gets the current 
state for the session and gets the next state for it based on the provided value.
The mapping is then updated with the generated state and the result from the current state
is passed down the pipeline.

When a session reaches a Terminal state, the session id is removed from the map
and the final result is outputted.

The states in the mapping are timestamped and are removed from the map after
the session expiration time, if they haven't finished before that.

% More hype
The abstraction using states allows the users of the package to implement the
logic for the session, independently of the session management part.

\subsection{Caching}
\texttt{CacheComponent} is a Component generator with the following signature:
\begin{figure}[h]
\centering
\begin{lstlisting}
func CacheComponent(cache Storage, worker Component, 
            expiration time.Duration, useCleaner bool) Component
\end{lstlisting}
\caption[scale=1.0]{Type signature of the \texttt{Cache Component} generator.}
\label{fig:cacheComp}
\end{figure}

The returned component behaves as follows:
\begin{itemize}
	\item When a value is input that the component hasn't seen before, then
		  the value is passed to the provided worker and the result is stored in
		  a map.
	\item When the component gets a previously seen value, it returns the value
		  stored in the map, if it hasn't expired yet. If it expired, then the component
		  treats it as a new value.
	\item Expired values are regularly deleted from the map if the useCleaner
          option is set to \texttt{true}.
\end{itemize}

For the generated component to function properly, the worker must output
a value for every value that is sent to it. This is one of the conditions
that a component must meet, so this shouldn't be an issue.

\subsection{Dynamic Load Balancing}
\texttt{DynamicLoadBalancer} is a Component generator whose signature is
shown in Figure \ref{fig:loadComp} below. Its architecture is similar
to that of the \texttt{StaticLoadBalancer}.
\begin{figure}[h]
\centering
\begin{lstlisting}
func DynamicLoadBalancer(addTimeout, removeTimeout time.Duration, 
        worker Component, maxWorkers int) Component
\end{lstlisting}
\caption[scale=1.0]{Declaration of \texttt{DynamicLoadBalancing} 
component generator.}
\label{fig:loadComp}
\end{figure}

The returned component has an array of workers which can be easily shut down.
The initial number of workers is 1. 
For every input value, the component behaves as follows:
\begin{itemize}
	\item It tries to send the value to the workers, if this is not successful, 
		  before the \texttt{addTimeout}, then the component creates a new worker
          if the current number of workers is smaller than \texttt{maxWorkers}.
          Then the component tries to send the value to the workers again. This will 
          be successful in case a worker was added because there is at least one 
          worker that is not busy.

	\item The workers send their results further down the pipeline.

	\item If there are no incoming values for duration equal to the 
          \texttt{removeTimeout}, then the component shuts down one 
          of the workers, if there is more than one worker.

	\item When the input channel of the component is closed, then the load balancer 
          shuts down all the workers and afterwards closes its output channel and
          terminates.
\end{itemize}
There are a lot of other possible load balancing strategies. 
Implementing them should be a simple modification of the existing code.
Note that this is very powerful concept as it can also be used to share
the load between multiple servers and turn them on or shut them down as
required (here using a different strategy might be preferred). There is
similar dynamic load balancer for writers.

\subsection{Error handling}

\subsubsection{Error Passer}
As a lot of components expect an input of a certain type, their execution
might end with an error if they are provided with an error result from
a previous component. This would make debugging very hard. Hence, the 
package provides the following component wrapper, that won't pass errors
to the provided component. This signature of this wrapper is shown below.
\begin{figure}[h]
\centering
\begin{lstlisting}
func ErrorPasser(worker Component) Component
\end{lstlisting}
\caption[scale=1.0]{Declaration of Error Passer Component generator.}
\label{fig:ErrorPasser}
\end{figure}

The generated component behaves as follows:
\begin{itemize}
	\item For every input value it checks whether the result is an error
		  and if that is the cases it just outputs the value.
	\item If the result of the input value is not an error, then the component
		  passes the value to the worker, which after processing it passes 
		  it further down the pipeline.
\end{itemize}

% TODO: finish the diagram
\begin{figure}[h]
\centering
\begin{tikzpicture}[node distance = 2cm, auto]
    % Nodes
    \node[component] (passer) {ErrorPasser};
    \node[component, right of=passer, node distance=3cm] (worker) {Worker};
    \node[right of=worker] (right) {};
    \node[left of=passer] (left) {};

    % Edges
    \path[line] (left) -- (passer);
    \path[line] (passer) -- (worker);
    \path[line] (worker) -- (right);
    % \path[line] (passer) -- (right);
\end{tikzpicture}
\caption[scale=1.0]{Inner design of the components generated with
  \texttt{ErrorPasser}.}
\label{fig:errPasserDiag}
\end{figure}

\subsubsection{Panic handling component}
If we use a component which might cause panic, this will crash the whole
server when the panic occurs. Hence, the package provides the 
\texttt{PanicHandlingComponent}. The signature of this wrapper is shown 
below.
\begin{figure}[h]
\centering
\begin{lstlisting}
func PanicHandlingComponent(worker Component) Component
\end{lstlisting}
\caption[scale=1.0]{Declaration of Panic Handling Component generator.}
\label{fig:panicHandler}
\end{figure}

It takes a component that can cause panic and returns a component the 
behaves as follows:
\begin{itemize}
	\item It passes every input value to the worker, gets the result 
          from it and then passes the result further down the pipeline.
	\item In case the worker crashes, that is causes panic, the component 
          writes an error as a result for the value that caused the crash 
          and the restarts the worker.
	\item The component can itself cause a panic if the worker closes its 
          input channel, or if it closes its output channel before its input 
          channel is closed. That is if the worker violates the contract
          for components.
\end{itemize}

\subsection{Network Components}
In this section I present a component generator that can make network
requests. Then I introduce a helper component that alters incoming 
requests, so that they can be performed again using a different host
and a component that processes HTTP responses. Finally, I show how
can these components be plugged together to act as a proxy server. 

\subsubsection{Network Component}
Network Component is a component generator with the following signature:
\begin{figure}[h]
\centering
\begin{lstlisting}
func NetworkComponent(client *http.Client) Component
\end{lstlisting}
\caption[scale=1.0]{Declaration of Network Component generator.}
\label{fig:NetworkComponent}
\end{figure}

It takes a single parameter, which is an http client.
The generated component for each input value makes an http request,
taken from the Result field of the value, using the provided http client.
If there is no request in the Result field or the request fails, then
the component returns an appropriate error. The behavior of the generated 
component is type described in Figure \ref{fig:networkComp}
below.

\begin{figure}[h]
\centering
\begin{tikzpicture}[node distance = 2cm, auto]
    % Nodes
    \node [component, densely dotted, ellipse] (internet) {Network};
    \node [component, text width=6em, below of=internet, node distance=3cm] (network) {Network Component};
    \node [left of=network, node distance = 3cm] (in) {};
    \node [right of=network, node distance = 3cm] (out) {};
    % Edges
    \path[line] (network) edge [bend right] (internet);
    \path[line] (internet) edge [bend right] (network);
    \path[line] (in) edge node {in} (network);
    \path[line] (network) edge node {out} (out);
\end{tikzpicture}
\caption[scale=1.0]{Design of the network component.}
\label{fig:networkComp}
\end{figure}

Note that the result of the request is fetched in the goroutine where the
component runs. Hence, slow requests might block the pipeline. To avoid
this behavior the \texttt{NetworkComponent} is usually used together with one
of the load managers.

\subsubsection{Request Copier}
Request Copier is a component generator with the following signature:
\begin{figure}[h]
\centering
\begin{lstlisting}
func RequestCopier(scheme, host string) Component
\end{lstlisting}
\caption[scale=1.0]{Declaration of Request Copier Component generator.}
\label{fig:RequestCopier}
\end{figure}
It takes string parameters scheme (e.g. http) and host (e.g. www.google.com) 
and returns a component that for each input value copies the Request made 
by the client to the Result field and updates it with the given scheme and 
path and then outputs this value.

\subsubsection{Response Processor}
The \texttt{Response} type of the mpserver package is defined as a following struct:

\begin{figure}[h]
\centering
\begin{lstlisting}
type Response struct {
		Header http.Header
		Body []byte
}
\end{lstlisting}
\caption[scale=1.0]{Declaration of type Response.}
\label{fig:Response}
\end{figure}

It holds the header and body of a http response. The difference between
this definition and the definition in the default http package is that
the body of my response is a slice of bytes. That is the body have already been
read and the connection to server has been closed.

The Response Processor is a component that for each input value
looks if provided Result is the default http Response, and if it is
it then reads it and transfers it to mpsever Response type, which it
then writes to the Result field and outputs the value. If no response
is provided, then the component outputs an appropriate error.

To use this component after the Network component, the Response Processor
should be wrapped in an error passing component, so that when the 
Network component fails, then the response processor won't have to do
any work.

\subsubsection{Proxy Component}
Proxy component is a combination of \texttt{RequestCopier}, 
\texttt{NetworkComponent} 
and \texttt{ResponseProcessor} in a linear pipeline. It acts as a proxy server
in the following sense. It forwards the provided request to the specified
host and then reads and outputs the response. 

All components are wrapped
in an \texttt{ErrorPasser} to avoid doing unnecessary work and to make the component
more robust. Furthermore, the linear combination of all the components is wrapped in 
\texttt{DynamicLoadBalancer} to make the whole component more 
responsive. The implementation is shown below:

\begin{figure}[h]
\centering
\begin{lstlisting}
func ProxyComponent(scheme, host string, client *http.Client, 
        addTime, removeTime time.Duration, nReq int) Component {
    return DynamicLoadBalancer(
        addTimeout, removeTimeout,
        LinkComponents(
            ErrorPasser(RequestCopier(scheme, host)),
            ErrorPasser(NetworkComponent(client)),
            ErrorPasser(ResponseProcessor)),
        nReq)
}
\end{lstlisting}
\caption[scale=1.0]{Declaration of Proxy Component generator.}
\label{fig:ProxyComp}
\end{figure}
\newpage
\subsection{Summary}
In this section I introduced more advanced component wrappers and generators
and described a way to use them. The next section shows a few example
servers implemented using my toolkit.

